% DO NOT COMPILE THIS FILE DIRECTLY!
% This is included by the other .tex files.

\begin{frame}[t,plain]
\titlepage
\end{frame}

\begin{frame}[t]{Motivating Problem I}
\begin{itemize}
\item How many sinusoids are in this data?
\end{itemize}
\begin{center}
\includegraphics[scale=0.35]{sinewave_data.pdf}
\end{center}
\begin{itemize}
\item Also, what are their periods, amplitudes and phases?
\end{itemize}
\end{frame}

\begin{frame}[t]{Motivating Problem II}
\begin{itemize}
\item How many galaxies are in this data?
\end{itemize}
\begin{center}
\includegraphics[scale=0.35]{../Paper/galaxyfield_data.pdf}
\end{center}
\begin{itemize}
\item Also, what are their positions, sizes, orientations, etc?
\end{itemize}
\end{frame}


\begin{frame}[t]{Context}
\begin{itemize}
\item Many problems have the following structure:
  \begin{itemize}
  \item There are $N$ objects in a region, and we don't know the value of $N$
  \item Each object has properties $\mathbf{x}_i$
  \item We have some data $\mathcal{D}$ which we want to use to infer both $N$
        and $\{\mathbf{x}_i\}$.
  \end{itemize}
\end{itemize}
\end{frame}



\begin{frame}[t]{Bayesian Inference 101: I}
Bayesian Inference is a unified framework for solving inference problems.
We need the following ingredients:

\begin{itemize}
\item A ``hypothesis space'' describing the set of possible answers to our
question (``parameter space'' in fitting is the same concept).
\item A ``prior distribution'' $\pi\left(\theta\right)$ describing how plausible
each of the possible solutions is, not taking into account the data.
\end{itemize}

\end{frame}


\begin{frame}[t]{Bayesian Inference 101: II}

The data helps us by changing our prior distribution into the ``posterior''
distribution, given by
\begin{eqnarray}
\frac{\pi(\theta)\mathcal{L}(\theta)}{\mathcal{Z}}
\end{eqnarray}
\end{frame}




