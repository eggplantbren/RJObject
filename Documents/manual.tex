% Every Latex document starts with a documentclass command
\documentclass[a4paper, 11pt]{article}

% Load some packages
\usepackage{graphicx} % This allows you to put figures in
\usepackage{natbib}   % This allows for relatively pain-free reference lists
\usepackage[left=2cm,top=3cm,right=2cm]{geometry} % The way I like the margins
\usepackage{framed}

\newcommand{\rjobject}{{\tt RJObject}}
\newcommand{\dnest}{{\tt DNest3}}

% This helps with figure placement
\renewcommand{\topfraction}{0.85}
\renewcommand{\textfraction}{0.1}
\parindent=0cm

% Set values so you can have a title
\title{RJObject Manual}
\author{Brendon J. Brewer}
\date{\today}

% Document starts here
\begin{document}

% Actually put the title in
\maketitle

\section{Introduction}
RJObject is a C++ library for implementing certain kinds of trans-dimensional
hierarchical models (basically `mixture models')in DNest3. I developed this
library after realising that many models I was implementing all had this
common structure.
Throughout this manual, I will assume you're proficient at
DNest3. If not, please read the manual for that project first.

\section{Dependencies}
As well as all the DNest3 dependencies, it's best if you add some of DNest3's
directories to your various PATH variables. Specifically, things will work best
if you put the location of the DNest3 libraries (the location of
the {\tt libdnest3.so} file) to your {\tt LIBRARY\_PATH} and
{\tt LD\_LIBRARY\_PATH}. Also add the directory containing the DNest3 header
files ({\tt *.h}) to your {\tt CPLUS\_INCLUDE\_PATH}. Finally, add the
directory containing DNest3's {\tt postprocess.py} and {\tt dnest\_plots.py} to
your {\tt PYTHON\_PATH}.

\end{document}

